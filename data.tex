\documentclass[a4paper,12pt]{jsreport}
% jsarticleは短めの論文につかう。jsreportは長めの論文
\usepackage{bm}
\usepackage[dvipdfmx]{graphicx}%図
\usepackage{ascmac}
\usepackage{bm}
\usepackage{amsmath}
% DEBUG:このlistingsが読み込み失敗している
% \usepackage{listings, plistings}% ソースコード用
\usepackage{listings}% ソースコード用
\usepackage{xcolor}

\lstset{
  % language=JavaScript,        % 言語の指定
  % basicstyle=\ttfamily\small, % 基本的なフォントスタイル
  % keywordstyle=\color{blue},  % キーワードの色
  stringstyle=\color{red},    % 文字列の色
  % commentstyle=\color{green}, % コメントの色
  % morecomment=[l][\color{magenta}]{\#},
  % breaklines=true,            % 長い行を折り返す
  % numbers=left,               % 行番号の位置
  % numberstyle=\tiny,          % 行番号のスタイル
  % frame=single,               % 枠線のスタイル
  % tabsize=4                   % タブのサイズ
}
% \usepackage{subfiles}

% ここでAPAのテスト
% \usepackage[backend=biber, style=apa]{biblatex}
% \addbibresource{reference.bib}

% このdocumentclassからbeginまでを、プリアンブルと呼ぶ
\includeonly{
  introduction.tex,
  % ./part2/chapter1.tex
  appendix.tex
  } 
\begin{document}
\tableofcontents

% :::タイトル
\title{身体運動の創造的学びの一人称研究}
\author{堀内隆仁}
\maketitle

% :::アブスト
\section*{概要}
本研究は、身体運動表現の創造的な学習過程にかんする一人称研究である。
「生きる」身体運動実践家は、その渦のなかで、その我が身でもって、表現をめざす。
従来の身体運動表現の研究は、科学的に精密に記述(客観至上主義)されてきた。
心理を扱う研究群も、「事実=もの=リアリティ」として、ちっぽけな「主観」の記述にとどまってきた。
欠如してきたのは、「現実=こと=アクチュアリティ」としての側面である。
私たちが生きているのは「現実」であって、それを「事実」に仮託するのは間違いである。
現象学・身体性認知科学・生態心理学・ゲシュタルト心理学・プラグマティズム哲学をはじめとして、
「生きている」ひとの知性の側面が
本研究では、現場で実践しながら進行形の創造プロセスとして現実を描き出す、一人称研究をとった。
これまで描かれてこなかった、身体運動表現の学び(創造プロセスとして)を描いた。

第一部は、著者自身が陸上競技(十種競技)の実践家として、「走り」という身体スキルを自らの現実を生きながら培ったプロセスを物語るものである。
はじめ、全身各部位や地面とに成り立つ多種多様な関係性へ志向していった。
ある怪我をきっかけとして、次第に、著者は自らの日常生活の動きや生活を取り巻くモノ(洗濯機や石ころやリュック)さえもが、
自らのスキルと密接に関わっていることを自覚し、実践に巻き込んでいった。
この過程にともなって、著者の走りの動作にも変化がみられた。

第二部では、実践家が「動く身体」を観るときに、その身体が醸し出す「表情」(=身体表情≠身体運動)を感得することを支援する
アプリケーションを制作した。感得した身体表情こそが、創造の源になり、実践家の学びを大いに促進しうるという仮説にもとづいている。
制作したアプリケーションをもちいて、学習実践をおこなった。
このアプリの設計には、第一部で著者自身が営んだスキル学習によって培ったものごとを色濃く反映している。

両部ともに一人称研究の方法論をとっている。
パラダイム転換しようと目論む。
随所で、神経科学的知見を参照し、その論拠としてもちいる。
現象学的な考え方をふまえて、いまだ認知についての学問領域には十分検討されていない「表情」をめぐる問題系について、
アプリとして「計算可能性と不可能性」を探究したことに価値がある。
現象学では通常、言葉をもちいて論じたり語る方法論をとる。
本研究のアプリで映像表現したりすることで、言葉だけでは迫り得なかったアクチュアリティに迫ることができた。

% \section*{概要}
本研究は、身体運動表現の創造的な学習過程にかんする一人称研究である。
「生きる」身体運動実践家は、その渦のなかで、その我が身でもって、表現をめざす。
従来の身体運動表現の研究は、科学的に精密に記述(客観至上主義)されてきた。
心理を扱う研究群も、「事実=もの=リアリティ」として、ちっぽけな「主観」の記述にとどまってきた。
欠如してきたのは、「現実=こと=アクチュアリティ」としての側面である。
私たちが生きているのは「現実」であって、それを「事実」に仮託するのは間違いである。
現象学・身体性認知科学・生態心理学・ゲシュタルト心理学・プラグマティズム哲学をはじめとして、
「生きている」ひとの知性の側面が
本研究では、現場で実践しながら進行形の創造プロセスとして現実を描き出す、一人称研究をとった。
これまで描かれてこなかった、身体運動表現の学び(創造プロセスとして)を描いた。

第一部は、著者自身が陸上競技(十種競技)の実践家として、「走り」という身体スキルを自らの現実を生きながら培ったプロセスを物語るものである。
はじめ、全身各部位や地面とに成り立つ多種多様な関係性へ志向していった。
ある怪我をきっかけとして、次第に、著者は自らの日常生活の動きや生活を取り巻くモノ(洗濯機や石ころやリュック)さえもが、
自らのスキルと密接に関わっていることを自覚し、実践に巻き込んでいった。
この過程にともなって、著者の走りの動作にも変化がみられた。

第二部では、実践家が「動く身体」を観るときに、その身体が醸し出す「表情」(=身体表情≠身体運動)を感得することを支援する
アプリケーションを制作した。感得した身体表情こそが、創造の源になり、実践家の学びを大いに促進しうるという仮説にもとづいている。
制作したアプリケーションをもちいて、学習実践をおこなった。
このアプリの設計には、第一部で著者自身が営んだスキル学習によって培ったものごとを色濃く反映している。

両部ともに一人称研究の方法論をとっている。
パラダイム転換しようと目論む。
随所で、神経科学的知見を参照し、その論拠としてもちいる。
現象学的な考え方をふまえて、いまだ認知についての学問領域には十分検討されていない「表情」をめぐる問題系について、
アプリとして「計算可能性と不可能性」を探究したことに価値がある。
現象学では通常、言葉をもちいて論じたり語る方法論をとる。
本研究のアプリで映像表現したりすることで、言葉だけでは迫り得なかったアクチュアリティに迫ることができた。


% :::はじめに
\part*{はじめに}
\addcontentsline{toc}{part}{はじめに}
\chapter{研究背景}
本研究は第一部第二部からなる。両方の部に通底する風呂敷を広げておこう。
なんやかんあy。
\section{身体知ということば}
身体びた知、身体の知、身体が知っていること、、、。身体知の統一的な定義はいまだない。謎に満ちた概念である。
身体知というものを学ぶのか?学びという営みがそもそも身体知なのか?じゃあ「身体知の学び」とはなにを指すのか?

身体知の説明にもっとも頻繁に持ちだされる例は、「自転車の乗り方」であるように思う。たしかに自転車の乗り方は身体知である。
そして、「スポーツでアスリートの運動?は身体知」という言説も、私たちはとくに抵抗なく受け入れてしまうわけである。
自転車でもスポーツでも、身体知の例題となることに疑いはない。
だが、それらの例だけで考えるのが、ある種の「短絡」へ導いてしまっているようにも思えてならない。
「スポーツだから身体知」と受け入れたときに、、じゃあその身体知ってなにを指してるの?と突っ込まれると、口篭ってしまう。




いっぱい使われ方がある。古くはギリシャ哲学や東洋哲学の流れにも、私たちの知性の本質を身体に求めようとする流れはあった。
ひとつの大きな転換点は、1980年代ころだろう。当時エキスパートシステムによって栄えていた人工知能研究が、おおきな壁にぶち当たった。
フレーム問題である。そのころ、エキスパートシステムにはない「身体」という存在こそが、人間知性の本質ではないかということで、
ひとの知能・知性を探究するさまざまな学問領域で、身体の知の探究が活発になった。
あ、あ

本研究が軸足をおくのは、身体をめぐるさまざまな流れの交流点である。
身体知の「しくみ」を記述する関連研究は、生態心理学、神経科学、運動生理学、
身体知の「ありよう」を記述する関連研究は、身体に重きをおいた現象学、Somatics、Somaesthetic、「体験過程」の研究群、わざ、・・・。
その意味で本研究は、「この幹のこの太い枝から分かれたこの枝の、さらに分かれたこの細い枝の先端を進めます」というものではない。

\section{用語の定義づけ}
そういうわけから、本研究では、さまざまな用語をもちいることにする。
本性では、学際研究としてのにあるということを強くして本章では、それらを道具立てとして、定義しておく。

\begin{description}
  \item [身体立てる]\mbox{}\\
  身体知のもっとも広い意味での言い換えである。  
  本研究での身体知とは、私たちがものごとを認知するとは、いまだ不定のXを、「なにか」として身体立てることであると定義するのだ。  
  \item[流転する]\mbox{}\\
  あるものX「変化のなかにある」ことを、「Xが流転する」と表現する。あるものAがAとして在ることができるのは、たえまなく「変化」のなかに晒されているのだ。。
  よおお。。「変化」と表現すると、ものがAとして盤石にあって、
  
\end{description}

\section{本論文における言葉遣い}
本論文では、必要があれば、積極的に「メタファ」や「」をももちいる。
なぜか。知を身体知とみる本研究では、論述する(議論する)という人間の知も、身体知とみなすことになる。
身体知としての論述は、「(読者や著者の)身体から地続き」になるように展開すべきものであると考える。
尼ヶ崎\cite{amagasaki}によれば、メタファは、身体とつながった言葉の表現方法である。
科学論文では、メタファの使用は御法度とされている。
メタファは「詩的」であり、それは「論理的」であることと相反するという考え方にもとづくのだろう。
だが、それは頭知パースペクティブからみた結果にすぎないと著者は考える。

読者に「身体でわかる」ということを促すように説明するには、論理的説明がいつでも最適とは限らない。
むしろ、すべてを論理的に説明しようとしすぎると、身体性が欠落した説明にもなりかねない。
本論文がそうなってしまっては、「主張する形式(文章の書き方)」が「主張している内容(知は身体知である)」と自己矛盾していることにもなる。


このうえであらためて「論理的」という言葉を見つめ直してみると、「ひとつ一つのことがらを地続きにつなげる」という意味があることに気づく。
メタファの活用は、「対象のことがらと身体とを地続きにする」のならば、それは「論理的」であることとも密接に関係している
(詩的であることと論理的であることは相反というわけでもない、ということになる)。
くわえるに、「が高い」とか「ーー」とか、私たちがあまり疑いなく論理的表現として使っている言葉も、もともとはメタファだったという事実もある(リクール)。
\footnote{こういう表現を、リクールは「死んだ隠喩」と呼んでいる}。

以上の理由から、本研究では、論理のなかに適切箇所ではメタファを組み入れることで\footnote{メタファをもちいているときは、そうであると明示する}、読者に「身体でわかる」ことを促したい。

知の原理がちがうのだから。

また同様の考え方にもとづき、必要があれば「冗長な説明」もいとわない。
現象学的なこと
これは、システムのしくみを説明する章でも、(丁寧すぎる説明は、なるべく脚注にまわすが)


\section{本論文の構成}







% % :::第一部
\part{身体運動の創造的学びの物語}
\setcounter{chapter}{0} % 章番号をリセット
\chapter{背景:学びを描くということ}

\section{Gendlinのexperiencing, focusing}



\section{いいい}
堀内\cite{horiuchi_suwa:2020}は、
廣松渉\cite{hiromatsu:1989}は
尼ヶ崎\cite{amagasaki}は\cite{kimura}
しかしここで、「能動性」の観点から忘れてはならないのが、「気持つ」という心の作用である。
注意(志向)や気分や情や欲といったものが、知覚と行為のカップリングに、「作用」してくるのだ。
よく知られたものに、「カクテルパーティー効果」がある。ほかにもダックアンドラビットや
といったゲシュタルト心理学の知見の蓄積を参照すれば、「気持つ」が知覚に「作用」しうることは明らかだ
\footnote{
  「AがBに作用する」は、「AとBがカップリングにある」と微妙に異なるありようとして書き分けている。
  知覚と行為はともに、物的世界と心的世界の境界に生起する記号現象だが、
  気持つは心的世界での面が強い(もちろん厳密には、気持つというのも、興奮して汗が出るとか怖くて鳥肌が立つなど、両世界の境界に生起するものではある。)
}。
この能動性は、能動性とはいえど、「表象」という存在の前には、\textgt{ボトムアップ能動性}とも呼びうるものだ。
さしずめ表象は\textgt{トップダウンな能動性}と呼んでもよいだろう。
つまり、知覚はトップダウンとボトムアップ両方向の能動性によって成り立つものだと主張したい。
知覚・行為・思考からなる渦、とはこういうことである。






% % :::第二部
\part{身体の「表情」感得を促すアプリの制作と、その実践}
\setcounter{chapter}{0} % 章番号をリセット
\chapter{背景:身体の醸しだす表情}

\section{動く身体を鑑賞するとき、なにをどうみるのがよいのか?}
\label{sec:学びにおける、動く身体を鑑賞するという体験}
第一部でも表れているように、実践家は学ぶときに、自身の運動を撮影した動画や、他者の運動をみたり、
あるいはYouTubeで視聴したりする。

繰り返すが、実践家の学びとは、SOMA的な、一人称視点から「感じる」身体を研ぎ澄ますのである。
そのためには、その「動く身体」になにをどうみるのが良いのか?
こういう問題意識を、私(著者)は培ってきたのだった。
一人称研究として、これをデザインするようになっていた。
付録にまわすが、物語のもとになった期間のあと、そーゆープログラミング的なことをはじめるようになったのだ。

動画視聴というときに、もっともありがちなのは、「三人称的な身体」をみてしまうということである。
これは、ーーー。であり、スポーツバイオメカニクスの考え方とも直結する。
むろん、こうした見方が果たす意義も大きい。前述したようにーーーだからだ。
そして、実践家は(私たちは)、そうした態度をとるのがやりやすいというのもまた事実である。

これは主に、「視覚」という知覚の性格と密接に関連しているのだろう。
中村\cite{nakamura:2000}は、
視覚という知覚は、ほかの諸知覚から「独走」して、私たちの「明晰的な意識」と結びつきやすいことを指摘している。
近代科学的なものごとの捉え方もこの最たる例である。
だが、視覚をはじめとした諸知覚はほんらい、体性感覚を中心にして互いに統合されて「共通感覚」として成り立っていると中村は言う。

身体運動も視覚独走型に、明晰的な部分志向となってしまうというわけだ。

共通感覚的にみるということは、視覚だけではなく、みているんだけれども、見ているわたしの身体にも感覚が生じるということである。
ちょうどミラーニューロンシステムはこのことを支持する。
ここで日常の経験をつぶさに顧みてみると、たしかに、ーーやーーということがある。

そういう見方のほうが、「一人称視点から感じる身体」とはリンクしているように考えられもする。
少なくとも、みているわたしの「身体」がじかに関わっている見方ということになる。
では、そういう見方とはいったいなんなのか?


どう問題か?
では、そこにためには、一体なにをどう見れば良いのか?

\subsection{現象学身体学から}
金子らは、潜性的にーーーーということを主張する。’

\section{『表情』}
\subsection{表情論導入}
この問いによく答えるための重大なヒントが、哲学者・廣松渉\cite{hiromatsu:1989}の提唱した「表情」という概念にある。
表情とは、「顔の表情」をメタフォリカルに拡張した概念である。
一言で問いに答えるなら、「実践家は身体の表情をみるべき」となるのだが、
その理解のために、まずは表情について本節でじっくりと説明する。
表情とはなんなのか。
廣松は説いている。
\begin{quotation}
  風景に眼を向けて見よう。われわれの日常如実の体験相においては、いま例えば、
  「いま裏山の松の樹はガッシリとしているが大枝はノタウッテいる。
  崖にかけて淡竹がスクスクと伸びており、刃先はピンと張っている。
  ・・・小川はサラサラと流れ、魚はスイスイと泳いでいる。
  雪がヒラヒラと舞い始め、やがてシズシズと降りしきる。
  松はコンモリと雪帽子を被り、いよいよドッシリと落ち付いて見える。
  一陣の風がサッと捲き起こり、雪がパッと散る。が、松はカタクナに立っている。
  竹はタワワに軋み、雀がピョンピョンと枝渡りすると、ドタドタと雪が零れる。
  夕陽がノンビリと傾き、月影がソッと忍び寄って来る・・・。」

  環界的情景は、表情性に満ち充ちている。\\
  --- 出典: 廣松渉, 『表情』, 弘文堂, p.9
\end{quotation}
味わい深い語りであろう。カタカナで表記された部分が表情に相当する。
末尾の「環界的情景」とは、「私たちに生きられ・体験された、情景としての環境・世界」のことだと解釈してよいだろう
\footnote{
  廣松渉はこのように、「廣松語」とも呼びうるような、さまざまな独自の用語をもちいて論述をしており、その多くは、彼による解説が与えられていない。
  廣松語が廣松らしさを味わうスパイスになっていることは言うまでもないが、
  一見してその意味するところがつかみにくい語もある。
  本論文では、こうした廣松語が登場するたびに解説をほどこすことにする。
}。
廣松は続ける。

\begin{quotation}
  直接的な体験意識に即するとき、事物(というものが在って、それ)が表情性を帯びている、 
  という表現方式は実態には合わない。
  右の文章では、松がグネグネしているとか、淡竹がスクスク伸びているとか、
  事物的分節体が表情性を呈するかのような表現方式になっているが、
  現実的にはむしろ、グネグネしているあれ、スクスク伸びているこれの覚知が先であって、
  その覚知与件が松・竹として事物的に認知・命名されるというのが実情であろう。 \\
  --- 出典: 廣松渉, 『表情』, 弘文堂, pp.9-10
\end{quotation}
私たちの生の体験では、(1)「グネグネ」のような表情こそをまず先に感得して、
そのあと事後的に、(2)その表情を「竹」のような事物的なものとして捉える(捉えなおす)のだ、
と廣松は説いているのだ。
そう、この順番は、私たちの常識と反するのだ。
廣松のこの主張は、常識と反するかなりラディカルな主張である。
私たちの常識では、(1)「竹」という物がまずあって、(2)私がそれを認識したときに「グネグネ」という解釈を与えている
のだと考えがちだからだ。
しかし、体験をよくよく振り返ってみれば、そういう体験があるということにも気付かされる。

\subsection{表情の定義のようなもの}
\cite{hiromatsu:1989}においては、表情とはなにかという定義然とした記述もいくつか見られる。
それらを掲載する。
\begin{quote}
  表情感得とは、情緒価と即応価とを内自化せる知覚的現認(p.79)
\end{quote}
\begin{quote}
  「表情感得」こそが、(中略)知・情・意の分化以前的な本源的体験相の原基(p.79)
\end{quote}
\begin{quote}
  知覚相・情緒価・即応価という三契機の各々が、質態値・度量値・趨勢値を内自化せしめた相で現前する(p.79)
\end{quote}
表情論で廣松は、さまざまな経験科学的なことにもふれちえる。

以上を踏まえると、さきほどの(1)(2)は単に「逆転」しているのではない。
表情は全体性なのである。

表情はある種のパースペクティブである。「一切の現相が悉く表情生を帯びて感得される(p.79)」と主張する。
廣松は以下のように書いている。
\begin{quotation}
  駄目押しするまでもなく、「感得される表情現相」というのは、決して人物や動物の顔面表情や身体的挙措表情には限られない。
  原基的な相においては、前章で概説した通り、一切の現相が\ruby{悉}{ことごと}く表情性を帯びて
  \footnote{鋭い読者は「Xが表情性を帯びる」という表現方式は表情にふさわしくないのでは?
  とツッコんだかもしれない。
  補足しておくが、廣松は、「表現の便宜上、以下では事物が表情性を帯びた相で現前するかのように記す方式をも辞せないようにしよう(『表情』, p.10)」と
  断りをいれている。
  }
  感得される。
  ---なるほど、現相のうちには、これというほどの感情価やこれというほどの即応価を帯びていないものもある。
  だが、その場合でも、表情価が端的に\ruby{零}{ゼロ}なのではない。よしんば零としか言いようのない“欠如態”の相にあるとしても、
  欠如態は(いわゆる“無色透明”が一種の色であるのと類比的に)それ自身、れっきとした価値態であることを忘れてはなるまい。\\
  (『表情』, p.79)
\end{quotation}

\subsection{「惣」が<立ち現れる>}
「物」が<ある>、「心」が<思う>。常識的に私たちは、この二段構えの構図において、自身の体験を理解しようとしてしまう。
この構図はもはや、体験を理解しようとするときの前提・出発点となっており、ふだん、私たちはそれじたいを疑うことはあまりない。
言い換えるなら、私たちはこの構図に「囚われている」。
私たちが表情のなんたるかを深く納得するには、この構図から脱出しなければならないと著者は考える。
\footnote{心身二元論の構図を「捨て去る」とまでは言っていない。それだけに囚われていては、「表情」に迫れないという問題意識である。}。

現象学者・大森荘蔵の一連の哲学「立ち現れ一元論」は、上記の構図をまるごとひっくり返そうとする。
この新しい構図における「惣」とは、「表情」に相当すると考えて差し支えない(本研究の範疇においては少なくとも)。



ただここで言っているのは、私たちが自身の体験を
\subsection{簡単な事例:色の近く}
視覚の原始的な例として、色の知覚を取り上げる。
色とは私たちの意識にのぼる「質」である。色そのものは、世界には存在しない。
光刺激は、波とみなせ、波長をもつ。人間は、およそ400〜800nmの光を知覚することができる。
「光波長」と「色」の対応があり、およそ、400nmが紫、500nmが緑、800nmが赤として、色のグラデーションとなる。
なぜこうなるか。受容器には錐体細胞がある。錐体細胞は、次のグラフのようになる。

つまり、マゼンダ色に対応する単色光は、世界には存在しない。いわば私たちが脳で作り出している色なのだ。
このように、色の知覚という原始的な例において、しかも処理の「上流」ですでに、
私たちは「世界をそのままコピー」してはいない。世界を構成しているのだ。

知覚相は、認識以前の世界ではない。世界の「姿形」という「質」を、作り出したものである。




\subsection{表情のまとめと図式化}
以上をもって、。
表情論を端的にまとめる。

\section{身体の「表情」}
表情論を踏まえれば、「身体」の表情があることになる。
これが、\ref{sec:学びにおける、動く身体を鑑賞するという体験}節
で述べたような「生々しさ」に相当するものだと本研究では考える。
一人称から感じる身体を研ぎ澄ます実践家は、鑑賞対象となる動く身体の「表情」を感得することが重要である。

動きのなかにある身体は、その観測に先立つ所与の「姿形」などもっていない。
そこには、無数の姿形が潜んでいる、と言い換えてもよい。
であれば、身体をそなえた鑑賞者である私が、姿形を与えるのだ。
私との共振的関係のさなかにおいて、その関係性そのものとして、立ち現れるゲシュタルト、
それが身体の表情である。


私たちが所与として受けいれがちな「解剖学的身体」も、実はそうした無数に潜在する姿形のたったひとつが顕在化したものにすぎない!
しかも解剖学的身体は、広松が表情でいうところの、色でいえば「無色透明」に相当するものである。
それ自体がのっぺらぼうというだけの、ひとつの表情なのである。

たとえば私の物語でいえば、こういう連続写真(動画)に対して、ーーということを感じたのである。
これは、一種の表情の感得がかかわっている。

ほかにも、たとえば、バキシリーズから、宮本武蔵の動きが。

\section{ツールづくりへの洞察}
ここまで実践と、哲学書読み漁りと、表情との出会いによって、
実装が固まったのだった。

たしかにそう考えてみれば、
EuphratesとISSEY MIYAKEによ映像作品『ISSEY MIYAKE A-POC INSIDE』\cite{sato_euphrates}
かなり似ている。「ファッションデザイナーのイキイキとした動きを如実に表現する」という手法としても作品。

ダンス教育家のラバンは、ーー正二十面体によってバレエのフォームを創造しようとした。

また、一連の作品『Strand Beast』も、この線で非常に興味深い。
風だけを動力として、木製の機構がただ作動しているだけである。
にもかかわらず、そこに不思議な「情」を感じてしまうのだえる。

こうした「最小限の見た目」でこそかえって如実に表現できるのはなぜなのか?
キネティックアーティスト・George Rickeyの弁を借りよう。



% :::おわりに


% :::参考文献
% 下のサイトにいろんなスタイルあり。どれかを選ぶ。
% https://mathlandscape.com/latex-bibstyles/

\bibliographystyle{jplain}%連番形式
% \bibliographystyle{acm}%ACM形式
% \bibliographystyle{apalike}
% \bibliographystyle{apa}
\bibliography{reference}
% \printbibliography

% :::付録
\part*{付録}
\addcontentsline{toc}{part}{付録}
% \chapter{物語についての付録}
\chapter{アプリについての付録}
\section{最近傍マーカの計算法}
コンピュータグラフィックスの処理課程(グラフィックスパイプラインもしくはレンダリングパイプライン)は、以下の流れである。
たとえば、ある3点があるとして説明する。
\begin{enumerate}
  \item モデリング:世界座標系において、対象の3次元位置座標を決める
  \item 視点変換:対象をカメラからの眺めに変換(世界座標系からカメラ座標系への並行移動)
  \item (カリング)
  \item クリッピング
  \item (ラスタライズ)
  \item 画面に出力
\end{enumerate}
なお、ベクトルはボールド体で表記する。
空間にある2つのマーカの3次元位置座標$\bm{p}$,$\bm{ q}$と、画面上のマウスカーソルの2次元座標$\bm{m}$の「距離」$d$を算出する。
マーカの位置をそれぞれ、
$$
  \bm{p} =
  \begin{pmatrix}
    {p_x} \\
    {p_y} \\
    {p_z}
  \end{pmatrix}
  ,
  \bm{q} =
  \begin{pmatrix}
    {q_x} \\
    {q_y} \\
    {q_z}
  \end{pmatrix}
  ,
  \bm{m} =
  \begin{pmatrix}
    {m_x} \\
    {m_y} \\
    {m_z}
  \end{pmatrix}
$$

モデル行列は今回は関係ない。
カメラの外部パラメータと内部パラメータによって、変換ができる。

カメラの外部パラメータは、以下3つからなる。
\begin{itemize}
  \item 世界座標系におけるカメラの位置(3次元ベクトル)
  \item 世界座標系におけるカメラの注視点(3次元ベクトル)
  \item 世界座標系におけるカメラの上方向(3次元ベクトル)
\end{itemize}
これら3つの情報は、1つの「ビュー行列」にまとめて表現できる。
$$
\mathbf{V} = 
\begin{pmatrix}
r_{11} & r_{12} & r_{13} & t_x \\
r_{21} & r_{22} & r_{23} & t_y \\
r_{31} & r_{32} & r_{33} & t_z \\
0 & 0 & 0 & 1
\end{pmatrix}
$$


カメラの内部パラメータは、以下4つ
\footnote{
  これら4つの情報をもちいて、図のように、カメラからみた世界全体から、ある「四角錐台」の領域を一意に定めている。
  この四角錐台の領域のみ、「カメラに映り込む世界」として切り取るわけだ。なぜ直方体ではなく四角錐台なのか?
  それは「カメラに近ければ近いほど大きく写り、遠ければ遠いほど小さく映る」ということを反映するためである。
  このあと四角錐台を、2つの底面を押しつぶすように、二次元平面へと圧縮するが、それで反映される。
}
からなる。
\begin{itemize}
  \item 焦点距離
  \item アスペクト比
  \item ニアクリップとファークリップ平面の距離
  \item 視野(Field of View, FoV)
\end{itemize}
これら4つの情報は、1つの「プロジェクション行列」にまとめて表現できる。
% ビュー行列をかける
$$
\mathbf{P} = 
\begin{pmatrix}
\frac{1}{\tan(\text{FOV}/2) \times \text{aspect}} & 0 & 0 & 0 \\
0 & \frac{1}{\tan(\text{FOV}/2)} & 0 & 0 \\
0 & 0 & \frac{f+n}{f-n} & \frac{2fn}{f-n} \\
0 & 0 & -1 & 0
\end{pmatrix}
$$

したがって、対象点の3次元ベクトルに、ビュー行列をかけてからプロジェクション行列をかけることで、
変換後の画面座標系で表された2次元ベクトル$\bm{p'} =
\begin{pmatrix}
  {p'_x} \\
  {p'_y} \\    
\end{pmatrix}
$が得られる。

$$
\bm{p'} =  
  \mathbf{P} \mathbf{V} \bm{p}
  =
  \begin{pmatrix}
    \frac{1}{\tan(\text{FoV}/2) \times \text{aspect}} & 0 & 0 & 0 \\
    0 & \frac{1}{\tan(\text{FoV}/2)} & 0 & 0 \\
    0 & 0 & \frac{f+n}{f-n} & \frac{2fn}{f-n} \\
    0 & 0 & -1 & 0
    \end{pmatrix}
    \begin{pmatrix}
      r_{11} & r_{12} & r_{13} & t_x \\
      r_{21} & r_{22} & r_{23} & t_y \\
      r_{31} & r_{32} & r_{33} & t_z \\
      0 & 0 & 0 & 1
    \end{pmatrix}    
    \begin{pmatrix}
    {p_x} \\
    {p_y} \\
    {p_z} \\
    {0}
    \end{pmatrix}
$$

すべてのマーカ(n個)に対して、これらの画面上2次元座標を求める。
すべてのマーカの画面上2次元座標とマウスカーソルとの距離をそれぞれ求める($d_1, d_2, ..., d_n$)。
あらかじめ画面2次元平面上でマウスカーソルを中心とした一定の半径$r$内にある点群を、「カーソル近傍マーカ」とするようにしておき、
$r > d$なる点群を求める。

「カーソル近傍マーカ」群のうちもっとも「カメラからみて手前」にあるものを、「最近傍マーカ」とする。



\section{開発にもちいたJavaScriptパッケージ}
\textcolor{red}{あかーーいテキストfeiji}
以下がnpmパッケージ一覧である。
% \begin{lstlisting}[basicstyle=\ttfamily\footnotesize, frame=single][htbp]
%   function hello(){
%     console.log('hello')
%   }
% \end{lstlisting}

\begin{lstlisting}
  function hello(){
    console.log('hello')
  }
\end{lstlisting}


  % {
  %   "name": "template",
  %   "version": "0.1.0",
  %   "private": true,
  %   "scripts": {
  %     "serve": "vue-cli-service serve",
  %     "build": "vue-cli-service build",
  %     "lint": "vue-cli-service lint"
  %   },
  %   "dependencies": {
  %     "@mediapipe/pose": "^0.5.1675469404",
  %     "core-js": "^3.8.3",
  %     "firebase": "^9.7.0",
  %     "firebase-admin": "^10.1.0",
  %     "hashids": "^2.3.0",
  %     "lodash": "^4.17.21",
  %     "matrixgl": "^2.0.0",
  %     "p5": "^1.4.1",
  %     "vue": "^2.6.14",
  %     "vue-router": "^3.5.1",
  %     "vuetify": "^2.6.0"
  %   },
  %   "devDependencies": {
  %     "@babel/core": "^7.12.16",
  %     "@babel/eslint-parser": "^7.12.16",
  %     "@vue/cli-plugin-babel": "~5.0.0",
  %     "@vue/cli-plugin-eslint": "~5.0.0",
  %     "@vue/cli-plugin-router": "~5.0.0",
  %     "@vue/cli-service": "~5.0.0",
  %     "eslint": "^7.32.0",
  %     "eslint-plugin-sort-imports-es6-autofix": "^0.6.0",
  %     "eslint-plugin-vue": "^8.0.3",
  %     "sass": "~1.32.0",
  %     "sass-loader": "^10.0.0",
  %     "vue-cli-plugin-vuetify": "~2.4.8",
  %     "vue-template-compiler": "^2.6.14",
  %     "vuetify-loader": "^1.7.0"
  %   }
  % }  



\end{document}