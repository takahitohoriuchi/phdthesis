\part*{付録}
\addcontentsline{toc}{part}{付録}
% \chapter{物語についての付録}
\chapter{アプリについての付録}
\section{最近傍マーカの計算法}
コンピュータグラフィックスの処理課程(グラフィックスパイプラインもしくはレンダリングパイプライン)は、以下の流れである。
たとえば、ある3点があるとして説明する。
\begin{enumerate}
  \item モデリング:世界座標系において、対象の3次元位置座標を決める
  \item 視点変換:対象をカメラからの眺めに変換(世界座標系からカメラ座標系への並行移動)
  \item (カリング)
  \item クリッピング
  \item (ラスタライズ)
  \item 画面に出力
\end{enumerate}
なお、ベクトルはボールド体で表記する。
空間にある2つのマーカの3次元位置座標$\bm{p}$,$\bm{ q}$と、画面上のマウスカーソルの2次元座標$\bm{m}$の「距離」$d$を算出する。
マーカの位置をそれぞれ、
$$
  \bm{p} =
  \begin{pmatrix}
    {p_x} \\
    {p_y} \\
    {p_z}
  \end{pmatrix}
  ,
  \bm{q} =
  \begin{pmatrix}
    {q_x} \\
    {q_y} \\
    {q_z}
  \end{pmatrix}
  ,
  \bm{m} =
  \begin{pmatrix}
    {m_x} \\
    {m_y} \\
    {m_z}
  \end{pmatrix}
$$

モデル行列は今回は関係ない。
カメラの外部パラメータと内部パラメータによって、変換ができる。

カメラの外部パラメータは、以下3つからなる。
\begin{itemize}
  \item 世界座標系におけるカメラの位置(3次元ベクトル)
  \item 世界座標系におけるカメラの注視点(3次元ベクトル)
  \item 世界座標系におけるカメラの上方向(3次元ベクトル)
\end{itemize}
これら3つの情報は、1つの「ビュー行列」にまとめて表現できる。
$$
\mathbf{V} = 
\begin{pmatrix}
r_{11} & r_{12} & r_{13} & t_x \\
r_{21} & r_{22} & r_{23} & t_y \\
r_{31} & r_{32} & r_{33} & t_z \\
0 & 0 & 0 & 1
\end{pmatrix}
$$


カメラの内部パラメータは、以下4つ
\footnote{
  これら4つの情報をもちいて、図のように、カメラからみた世界全体から、ある「四角錐台」の領域を一意に定めている。
  この四角錐台の領域のみ、「カメラに映り込む世界」として切り取るわけだ。なぜ直方体ではなく四角錐台なのか?
  それは「カメラに近ければ近いほど大きく写り、遠ければ遠いほど小さく映る」ということを反映するためである。
  このあと四角錐台を、2つの底面を押しつぶすように、二次元平面へと圧縮するが、それで反映される。
}
からなる。
\begin{itemize}
  \item 焦点距離
  \item アスペクト比
  \item ニアクリップとファークリップ平面の距離
  \item 視野(Field of View, FoV)
\end{itemize}
これら4つの情報は、1つの「プロジェクション行列」にまとめて表現できる。
% ビュー行列をかける
$$
\mathbf{P} = 
\begin{pmatrix}
\frac{1}{\tan(\text{FOV}/2) \times \text{aspect}} & 0 & 0 & 0 \\
0 & \frac{1}{\tan(\text{FOV}/2)} & 0 & 0 \\
0 & 0 & \frac{f+n}{f-n} & \frac{2fn}{f-n} \\
0 & 0 & -1 & 0
\end{pmatrix}
$$

したがって、対象点の3次元ベクトルに、ビュー行列をかけてからプロジェクション行列をかけることで、
変換後の画面座標系で表された2次元ベクトル$\bm{p'} =
\begin{pmatrix}
  {p'_x} \\
  {p'_y} \\    
\end{pmatrix}
$が得られる。

$$
\bm{p'} =  
  \mathbf{P} \mathbf{V} \bm{p}
  =
  \begin{pmatrix}
    \frac{1}{\tan(\text{FoV}/2) \times \text{aspect}} & 0 & 0 & 0 \\
    0 & \frac{1}{\tan(\text{FoV}/2)} & 0 & 0 \\
    0 & 0 & \frac{f+n}{f-n} & \frac{2fn}{f-n} \\
    0 & 0 & -1 & 0
    \end{pmatrix}
    \begin{pmatrix}
      r_{11} & r_{12} & r_{13} & t_x \\
      r_{21} & r_{22} & r_{23} & t_y \\
      r_{31} & r_{32} & r_{33} & t_z \\
      0 & 0 & 0 & 1
    \end{pmatrix}    
    \begin{pmatrix}
    {p_x} \\
    {p_y} \\
    {p_z} \\
    {0}
    \end{pmatrix}
$$

すべてのマーカ(n個)に対して、これらの画面上2次元座標を求める。
すべてのマーカの画面上2次元座標とマウスカーソルとの距離をそれぞれ求める($d_1, d_2, ..., d_n$)。
あらかじめ画面2次元平面上でマウスカーソルを中心とした一定の半径$r$内にある点群を、「カーソル近傍マーカ」とするようにしておき、
$r > d$なる点群を求める。

「カーソル近傍マーカ」群のうちもっとも「カメラからみて手前」にあるものを、「最近傍マーカ」とする。



\section{開発にもちいたJavaScriptパッケージ}
\textcolor{red}{あかーーいテキストfeiji}
以下がnpmパッケージ一覧である。
% \begin{lstlisting}[basicstyle=\ttfamily\footnotesize, frame=single][htbp]
%   function hello(){
%     console.log('hello')
%   }
% \end{lstlisting}

\begin{lstlisting}
  function hello(){
    console.log('hello')
  }
\end{lstlisting}


  % {
  %   "name": "template",
  %   "version": "0.1.0",
  %   "private": true,
  %   "scripts": {
  %     "serve": "vue-cli-service serve",
  %     "build": "vue-cli-service build",
  %     "lint": "vue-cli-service lint"
  %   },
  %   "dependencies": {
  %     "@mediapipe/pose": "^0.5.1675469404",
  %     "core-js": "^3.8.3",
  %     "firebase": "^9.7.0",
  %     "firebase-admin": "^10.1.0",
  %     "hashids": "^2.3.0",
  %     "lodash": "^4.17.21",
  %     "matrixgl": "^2.0.0",
  %     "p5": "^1.4.1",
  %     "vue": "^2.6.14",
  %     "vue-router": "^3.5.1",
  %     "vuetify": "^2.6.0"
  %   },
  %   "devDependencies": {
  %     "@babel/core": "^7.12.16",
  %     "@babel/eslint-parser": "^7.12.16",
  %     "@vue/cli-plugin-babel": "~5.0.0",
  %     "@vue/cli-plugin-eslint": "~5.0.0",
  %     "@vue/cli-plugin-router": "~5.0.0",
  %     "@vue/cli-service": "~5.0.0",
  %     "eslint": "^7.32.0",
  %     "eslint-plugin-sort-imports-es6-autofix": "^0.6.0",
  %     "eslint-plugin-vue": "^8.0.3",
  %     "sass": "~1.32.0",
  %     "sass-loader": "^10.0.0",
  %     "vue-cli-plugin-vuetify": "~2.4.8",
  %     "vue-template-compiler": "^2.6.14",
  %     "vuetify-loader": "^1.7.0"
  %   }
  % }  

