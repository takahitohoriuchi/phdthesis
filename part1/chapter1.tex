\part{身体運動の創造的学びの物語}
\chapter{研究背景}

\section{背景}
とくに哲学に明るい読者のために、本研究の哲学的立場を表明しておく。
もちろんこれらについては、本稿の随所で、より具体的かつ詳細に説明はする。
本節では、まずはその骨子のみ示しておく。

\begin{itemize}
  \item 箇条書きなのだ
  \item 箇条書き
  \item 箇条書き
\end{itemize}

\section{マブラヴオルタネイティブ}
廣松渉\cite{廣松}は
しかしここで、「能動性」の観点から忘れてはならないのが、「気持つ」という心の作用である。
注意(志向)や気分や情や欲といったものが、知覚と行為のカップリングに、「作用」してくるのだ。
よく知られたものに、「カクテルパーティー効果」がある。ほかにもダックアンドラビットや
といったゲシュタルト心理学の知見の蓄積を参照すれば、「気持つ」が知覚に「作用」しうることは明らかだ
\footnote{
  「AがBに作用する」は、「AとBがカップリングにある」と微妙に異なるありようとして書き分けている。
  知覚と行為はともに、物的世界と心的世界の境界に生起する記号現象だが、
  気持つは心的世界での面が強い(もちろん厳密には、気持つというのも、興奮して汗が出るとか怖くて鳥肌が立つなど、両世界の境界に生起するものではある。)
}。
この能動性は、能動性とはいえど、「表象」という存在の前には、\textgt{ボトムアップ能動性}とも呼びうるものだ。
さしずめ表象は\textgt{トップダウンな能動性}と呼んでもよいだろう。
つまり、知覚はトップダウンとボトムアップ両方向の能動性によって成り立つものだと主張したい。
知覚・行為・思考からなる渦、とはこういうことである。

\section{色の知覚}
視覚の原始的な例として、色の知覚を取り上げる。
色とは私たちの意識にのぼる「質」である。色そのものは、世界には存在しない。
光刺激は、波とみなせ、波長をもつ。人間は、およそ400〜800nmの光を知覚することができる。
「光波長」と「色」の対応があり、およそ、400nmが紫、500nmが緑、800nmが赤として、色のグラデーションとなる。
なぜこうなるか。受容器には錐体細胞がある。錐体細胞は、次のグラフのようになる。

つまり、マゼンダ色に対応する単色光は、世界には存在しない。いわば私たちが脳で作り出している色なのだ。
このように、色の知覚という原始的な例において、しかも処理の「上流」ですでに、
私たちは「世界をそのままコピー」してはいない。世界を構成しているのだ。

知覚相は、認識以前の世界ではない。世界の「姿形」という「質」を、作り出したものである。



\chapter{学びの物語}

\section{読み方と書き方について}
本章では、著者自身をさす一人称を、「私」と称する。

随所で、その知見に関係するであろう、経験科学的知見を引用する。
科学に「帰着」する、科学のほうが偉いという考え方では決してないが、多面的・深い理解を促すためには、
既存の科学的知見との関連づけることは重要であろう。