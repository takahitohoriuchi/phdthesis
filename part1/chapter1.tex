\part{身体運動の創造的学びの物語}
\setcounter{chapter}{0} % 章番号をリセット
\chapter{背景:学びを描くということ}

\section{Gendlinのexperiencing, focusing}



\section{いいい}
堀内\cite{horiuchi_suwa:2020}は、
廣松渉\cite{hiromatsu:1989}は
尼ヶ崎\cite{amagasaki}は\cite{kimura}
しかしここで、「能動性」の観点から忘れてはならないのが、「気持つ」という心の作用である。
注意(志向)や気分や情や欲といったものが、知覚と行為のカップリングに、「作用」してくるのだ。
よく知られたものに、「カクテルパーティー効果」がある。ほかにもダックアンドラビットや
といったゲシュタルト心理学の知見の蓄積を参照すれば、「気持つ」が知覚に「作用」しうることは明らかだ
\footnote{
  「AがBに作用する」は、「AとBがカップリングにある」と微妙に異なるありようとして書き分けている。
  知覚と行為はともに、物的世界と心的世界の境界に生起する記号現象だが、
  気持つは心的世界での面が強い(もちろん厳密には、気持つというのも、興奮して汗が出るとか怖くて鳥肌が立つなど、両世界の境界に生起するものではある。)
}。
この能動性は、能動性とはいえど、「表象」という存在の前には、\textgt{ボトムアップ能動性}とも呼びうるものだ。
さしずめ表象は\textgt{トップダウンな能動性}と呼んでもよいだろう。
つまり、知覚はトップダウンとボトムアップ両方向の能動性によって成り立つものだと主張したい。
知覚・行為・思考からなる渦、とはこういうことである。



