\part{身体の「表情」感得を促すアプリの制作と、その実践}
\setcounter{chapter}{0} % 章番号をリセット
\chapter{背景:身体の醸しだす表情}

\section{動く身体を鑑賞するとき、なにをどうみるのがよいのか?}
\label{sec:学びにおける、動く身体を鑑賞するという体験}
第一部でも表れているように、実践家は学ぶときに、自身の運動を撮影した動画や、他者の運動をみたり、
あるいはYouTubeで視聴したりする。

繰り返すが、実践家の学びとは、SOMA的な、一人称視点から「感じる」身体を研ぎ澄ますのである。
そのためには、その「動く身体」になにをどうみるのが良いのか?
こういう問題意識を、私(著者)は培ってきたのだった。
一人称研究として、これをデザインするようになっていた。
付録にまわすが、物語のもとになった期間のあと、そーゆープログラミング的なことをはじめるようになったのだ。

動画視聴というときに、もっともありがちなのは、「三人称的な身体」をみてしまうということである。
これは、ーーー。であり、スポーツバイオメカニクスの考え方とも直結する。
むろん、こうした見方が果たす意義も大きい。前述したようにーーーだからだ。
そして、実践家は(私たちは)、そうした態度をとるのがやりやすいというのもまた事実である。

これは主に、「視覚」という知覚の性格と密接に関連しているのだろう。
中村\cite{nakamura:2000}は、
視覚という知覚は、ほかの諸知覚から「独走」して、私たちの「明晰的な意識」と結びつきやすいことを指摘している。
近代科学的なものごとの捉え方もこの最たる例である。
だが、視覚をはじめとした諸知覚はほんらい、体性感覚を中心にして互いに統合されて「共通感覚」として成り立っていると中村は言う。

身体運動も視覚独走型に、明晰的な部分志向となってしまうというわけだ。

共通感覚的にみるということは、視覚だけではなく、みているんだけれども、見ているわたしの身体にも感覚が生じるということである。
ちょうどミラーニューロンシステムはこのことを支持する。
ここで日常の経験をつぶさに顧みてみると、たしかに、ーーやーーということがある。

そういう見方のほうが、「一人称視点から感じる身体」とはリンクしているように考えられもする。
少なくとも、みているわたしの「身体」がじかに関わっている見方ということになる。
では、そういう見方とはいったいなんなのか?


どう問題か?
では、そこにためには、一体なにをどう見れば良いのか?

\subsection{現象学身体学から}
金子らは、潜性的にーーーーということを主張する。’

\section{『表情』}
\subsection{表情論導入}
この問いによく答えるための重大なヒントが、哲学者・廣松渉\cite{hiromatsu:1989}の提唱した「表情」という概念にある。
表情とは、「顔の表情」をメタフォリカルに拡張した概念である。
一言で問いに答えるなら、「実践家は身体の表情をみるべき」となるのだが、
その理解のために、まずは表情について本節でじっくりと説明する。
表情とはなんなのか。
廣松は説いている。
\begin{quotation}
  風景に眼を向けて見よう。われわれの日常如実の体験相においては、いま例えば、
  「いま裏山の松の樹はガッシリとしているが大枝はノタウッテいる。
  崖にかけて淡竹がスクスクと伸びており、刃先はピンと張っている。
  ・・・小川はサラサラと流れ、魚はスイスイと泳いでいる。
  雪がヒラヒラと舞い始め、やがてシズシズと降りしきる。
  松はコンモリと雪帽子を被り、いよいよドッシリと落ち付いて見える。
  一陣の風がサッと捲き起こり、雪がパッと散る。が、松はカタクナに立っている。
  竹はタワワに軋み、雀がピョンピョンと枝渡りすると、ドタドタと雪が零れる。
  夕陽がノンビリと傾き、月影がソッと忍び寄って来る・・・。」

  環界的情景は、表情性に満ち充ちている。\\
  --- 出典: 廣松渉, 『表情』, 弘文堂, p.9
\end{quotation}
味わい深い語りであろう。カタカナで表記された部分が表情に相当する。
末尾の「環界的情景」とは、「私たちに生きられ・体験された、情景としての環境・世界」のことだと解釈してよいだろう
\footnote{
  廣松渉はこのように、「廣松語」とも呼びうるような、さまざまな独自の用語をもちいて論述をしており、その多くは、彼による解説が与えられていない。
  廣松語が廣松らしさを味わうスパイスになっていることは言うまでもないが、
  一見してその意味するところがつかみにくい語もある。
  本論文では、こうした廣松語が登場するたびに解説をほどこすことにする。
}。
廣松は続ける。

\begin{quotation}
  直接的な体験意識に即するとき、事物(というものが在って、それ)が表情性を帯びている、 
  という表現方式は実態には合わない。
  右の文章では、松がグネグネしているとか、淡竹がスクスク伸びているとか、
  事物的分節体が表情性を呈するかのような表現方式になっているが、
  現実的にはむしろ、グネグネしているあれ、スクスク伸びているこれの覚知が先であって、
  その覚知与件が松・竹として事物的に認知・命名されるというのが実情であろう。 \\
  --- 出典: 廣松渉, 『表情』, 弘文堂, pp.9-10
\end{quotation}
私たちの生の体験では、(1)「グネグネ」のような表情こそをまず先に感得して、
そのあと事後的に、(2)その表情を「竹」のような事物的なものとして捉える(捉えなおす)のだ、
と廣松は説いているのだ。
そう、この順番は、私たちの常識と反するのだ。
廣松のこの主張は、常識と反するかなりラディカルな主張である。
私たちの常識では、(1)「竹」という物がまずあって、(2)私がそれを認識したときに「グネグネ」という解釈を与えている
のだと考えがちだからだ。
しかし、体験をよくよく振り返ってみれば、そういう体験があるということにも気付かされる。

\subsection{表情の定義のようなもの}
\cite{hiromatsu:1989}においては、表情とはなにかという定義然とした記述もいくつか見られる。
それらを掲載する。
\begin{quote}
  表情感得とは、情緒価と即応価とを内自化せる知覚的現認(p.79)
\end{quote}
\begin{quote}
  「表情感得」こそが、(中略)知・情・意の分化以前的な本源的体験相の原基(p.79)
\end{quote}
\begin{quote}
  知覚相・情緒価・即応価という三契機の各々が、質態値・度量値・趨勢値を内自化せしめた相で現前する(p.79)
\end{quote}
表情論で廣松は、さまざまな経験科学的なことにもふれちえる。

以上を踏まえると、さきほどの(1)(2)は単に「逆転」しているのではない。
表情は全体性なのである。

表情はある種のパースペクティブである。「一切の現相が悉く表情生を帯びて感得される(p.79)」と主張する。
廣松は以下のように書いている。
\begin{quotation}
  駄目押しするまでもなく、「感得される表情現相」というのは、決して人物や動物の顔面表情や身体的挙措表情には限られない。
  原基的な相においては、前章で概説した通り、一切の現相が\ruby{悉}{ことごと}く表情性を帯びて
  \footnote{鋭い読者は「Xが表情性を帯びる」という表現方式は表情にふさわしくないのでは?
  とツッコんだかもしれない。
  補足しておくが、廣松は、「表現の便宜上、以下では事物が表情性を帯びた相で現前するかのように記す方式をも辞せないようにしよう(『表情』, p.10)」と
  断りをいれている。
  }
  感得される。
  ---なるほど、現相のうちには、これというほどの感情価やこれというほどの即応価を帯びていないものもある。
  だが、その場合でも、表情価が端的に\ruby{零}{ゼロ}なのではない。よしんば零としか言いようのない“欠如態”の相にあるとしても、
  欠如態は(いわゆる“無色透明”が一種の色であるのと類比的に)それ自身、れっきとした価値態であることを忘れてはなるまい。\\
  (『表情』, p.79)
\end{quotation}

\subsection{「惣」が<立ち現れる>}
「物」が<ある>、「心」が<思う>。常識的に私たちは、この二段構えの構図において、自身の体験を理解しようとしてしまう。
この構図はもはや、体験を理解しようとするときの前提・出発点となっており、ふだん、私たちはそれじたいを疑うことはあまりない。
言い換えるなら、私たちはこの構図に「囚われている」。
私たちが表情のなんたるかを深く納得するには、この構図から脱出しなければならないと著者は考える。
\footnote{心身二元論の構図を「捨て去る」とまでは言っていない。それだけに囚われていては、「表情」に迫れないという問題意識である。}。

現象学者・大森荘蔵の一連の哲学「立ち現れ一元論」は、上記の構図をまるごとひっくり返そうとする。
この新しい構図における「惣」とは、「表情」に相当すると考えて差し支えない(本研究の範疇においては少なくとも)。



ただここで言っているのは、私たちが自身の体験を
\subsection{簡単な事例:色の近く}
視覚の原始的な例として、色の知覚を取り上げる。
色とは私たちの意識にのぼる「質」である。色そのものは、世界には存在しない。
光刺激は、波とみなせ、波長をもつ。人間は、およそ400〜800nmの光を知覚することができる。
「光波長」と「色」の対応があり、およそ、400nmが紫、500nmが緑、800nmが赤として、色のグラデーションとなる。
なぜこうなるか。受容器には錐体細胞がある。錐体細胞は、次のグラフのようになる。

つまり、マゼンダ色に対応する単色光は、世界には存在しない。いわば私たちが脳で作り出している色なのだ。
このように、色の知覚という原始的な例において、しかも処理の「上流」ですでに、
私たちは「世界をそのままコピー」してはいない。世界を構成しているのだ。

知覚相は、認識以前の世界ではない。世界の「姿形」という「質」を、作り出したものである。




\subsection{表情のまとめと図式化}
以上をもって、。
表情論を端的にまとめる。

\section{身体の「表情」}
表情論を踏まえれば、「身体」の表情があることになる。
これが、\ref{sec:学びにおける、動く身体を鑑賞するという体験}節
で述べたような「生々しさ」に相当するものだと本研究では考える。
一人称から感じる身体を研ぎ澄ます実践家は、鑑賞対象となる動く身体の「表情」を感得することが重要である。

動きのなかにある身体は、その観測に先立つ所与の「姿形」などもっていない。
そこには、無数の姿形が潜んでいる、と言い換えてもよい。
であれば、身体をそなえた鑑賞者である私が、姿形を与えるのだ。
私との共振的関係のさなかにおいて、その関係性そのものとして、立ち現れるゲシュタルト、
それが身体の表情である。


私たちが所与として受けいれがちな「解剖学的身体」も、実はそうした無数に潜在する姿形のたったひとつが顕在化したものにすぎない!
しかも解剖学的身体は、広松が表情でいうところの、色でいえば「無色透明」に相当するものである。
それ自体がのっぺらぼうというだけの、ひとつの表情なのである。

たとえば私の物語でいえば、こういう連続写真(動画)に対して、ーーということを感じたのである。
これは、一種の表情の感得がかかわっている。

ほかにも、たとえば、バキシリーズから、宮本武蔵の動きが。

\section{ツールづくりへの洞察}
ここまで実践と、哲学書読み漁りと、表情との出会いによって、
実装が固まったのだった。

たしかにそう考えてみれば、
EuphratesとISSEY MIYAKEによ映像作品『ISSEY MIYAKE A-POC INSIDE』\cite{sato_euphrates}
かなり似ている。「ファッションデザイナーのイキイキとした動きを如実に表現する」という手法としても作品。

ダンス教育家のラバンは、ーー正二十面体によってバレエのフォームを創造しようとした。

また、一連の作品『Strand Beast』も、この線で非常に興味深い。
風だけを動力として、木製の機構がただ作動しているだけである。
にもかかわらず、そこに不思議な「情」を感じてしまうのだえる。

こうした「最小限の見た目」でこそかえって如実に表現できるのはなぜなのか?
キネティックアーティスト・George Rickeyの弁を借りよう。

