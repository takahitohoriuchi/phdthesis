\chapter{実践方法}

\section{対象実践家}
表題は有り体に言えば「被験者」のことだが、本研究では意図的に「被験者」という用語を使わない。
被験者ということばは、実験室実験の香りをまとっているように感じられるだろう。
対して本研究では、口酸っぱく言うように、実験室に閉ざされない、生きるということと向き合う。
したがって以降、「対象実践家」と呼ぶことにする。

5名の実践家を対象実践家とした。

モーションキャプチャによる撮影を実施した。
モーションキャプチャは、OptiTrack社製のV120:Trio(解像度30万画素、フレームレート120、レンズ視野角47°)をもちいた。
ソフトウェアはMotive(ver.2.1.0)である。LenovoPCでやった。