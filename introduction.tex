\part*{はじめに}
\addcontentsline{toc}{part}{はじめに}
\chapter{研究背景}
本研究は第一部第二部からなる。両方の部に通底する風呂敷を広げておこう。
なんやかんあy。
\section{身体知ということば}
身体びた知、身体の知、身体が知っていること、、、。身体知の統一的な定義はいまだない。謎に満ちた概念である。
身体知というものを学ぶのか?学びという営みがそもそも身体知なのか?じゃあ「身体知の学び」とはなにを指すのか?

身体知の説明にもっとも頻繁に持ちだされる例は、「自転車の乗り方」であるように思う。たしかに自転車の乗り方は身体知である。
そして、「スポーツでアスリートの運動?は身体知」という言説も、私たちはとくに抵抗なく受け入れてしまうわけである。
自転車でもスポーツでも、身体知の例題となることに疑いはない。
だが、それらの例だけで考えるのが、ある種の「短絡」へ導いてしまっているようにも思えてならない。
「スポーツだから身体知」と受け入れたときに、、じゃあその身体知ってなにを指してるの?と突っ込まれると、口篭ってしまう。




いっぱい使われ方がある。古くはギリシャ哲学や東洋哲学の流れにも、私たちの知性の本質を身体に求めようとする流れはあった。
ひとつの大きな転換点は、1980年代ころだろう。当時エキスパートシステムによって栄えていた人工知能研究が、おおきな壁にぶち当たった。
フレーム問題である。そのころ、エキスパートシステムにはない「身体」という存在こそが、人間知性の本質ではないかということで、
ひとの知能・知性を探究するさまざまな学問領域で、身体の知の探究が活発になった。
あ、あ

本研究が軸足をおくのは、身体をめぐるさまざまな流れの交流点である。
身体知の「しくみ」を記述する関連研究は、生態心理学、神経科学、運動生理学、
身体知の「ありよう」を記述する関連研究は、身体に重きをおいた現象学、Somatics、Somaesthetic、「体験過程」の研究群、わざ、・・・。
その意味で本研究は、「この幹のこの太い枝から分かれたこの枝の、さらに分かれたこの細い枝の先端を進めます」というものではない。

\section{用語の定義づけ}
そういうわけから、本研究では、さまざまな用語をもちいることにする。
本性では、学際研究としてのにあるということを強くして本章では、それらを道具立てとして、定義しておく。

\begin{description}
  \item [身体立てる]\mbox{}\\
  身体知のもっとも広い意味での言い換えである。  
  本研究での身体知とは、私たちがものごとを認知するとは、いまだ不定のXを、「なにか」として身体立てることであると定義するのだ。  
  \item[流転する]\mbox{}\\
  あるものX「変化のなかにある」ことを、「Xが流転する」と表現する。あるものAがAとして在ることができるのは、たえまなく「変化」のなかに晒されているのだ。。
  よおお。。「変化」と表現すると、ものがAとして盤石にあって、
  
\end{description}

\section{本論文における言葉遣い}
本論文では、必要があれば、積極的に「メタファ」や「」をももちいる。
なぜか。知を身体知とみる本研究では、論述する(議論する)という人間の知も、身体知とみなすことになる。
身体知としての論述は、「(読者や著者の)身体から地続き」になるように展開すべきものであると考える。
尼ヶ崎\cite{amagasaki}によれば、メタファは、身体とつながった言葉の表現方法である。
科学論文では、メタファの使用は御法度とされている。
メタファは「詩的」であり、それは「論理的」であることと相反するという考え方にもとづくのだろう。
だが、それは頭知パースペクティブからみた結果にすぎないと著者は考える。

読者に「身体でわかる」ということを促すように説明するには、論理的説明がいつでも最適とは限らない。
むしろ、すべてを論理的に説明しようとしすぎると、身体性が欠落した説明にもなりかねない。
本論文がそうなってしまっては、「主張する形式(文章の書き方)」が「主張している内容(知は身体知である)」と自己矛盾していることにもなる。


このうえであらためて「論理的」という言葉を見つめ直してみると、「ひとつ一つのことがらを地続きにつなげる」という意味があることに気づく。
メタファの活用は、「対象のことがらと身体とを地続きにする」のならば、それは「論理的」であることとも密接に関係している
(詩的であることと論理的であることは相反というわけでもない、ということになる)。
くわえるに、「が高い」とか「ーー」とか、私たちがあまり疑いなく論理的表現として使っている言葉も、もともとはメタファだったという事実もある(リクール)。
\footnote{こういう表現を、リクールは「死んだ隠喩」と呼んでいる}。

以上の理由から、本研究では、論理のなかに適切箇所ではメタファを組み入れることで\footnote{メタファをもちいているときは、そうであると明示する}、読者に「身体でわかる」ことを促したい。

知の原理がちがうのだから。

また同様の考え方にもとづき、必要があれば「冗長な説明」もいとわない。
現象学的なこと
これは、システムのしくみを説明する章でも、(丁寧すぎる説明は、なるべく脚注にまわすが)


\section{本論文の構成}





