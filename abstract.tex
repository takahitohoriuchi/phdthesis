\section*{概要}
本研究は、身体運動表現の創造的な学習過程にかんする一人称研究である。
「生きる」身体運動実践家は、その渦のなかで、その我が身でもって、表現をめざす。
従来の身体運動表現の研究は、科学的に精密に記述(客観至上主義)されてきた。
心理を扱う研究群も、「事実=もの=リアリティ」として、ちっぽけな「主観」の記述にとどまってきた。
欠如してきたのは、「現実=こと=アクチュアリティ」としての側面である。
私たちが生きているのは「現実」であって、それを「事実」に仮託するのは間違いである。
現象学・身体性認知科学・生態心理学・ゲシュタルト心理学・プラグマティズム哲学をはじめとして、
「生きている」ひとの知性の側面が
本研究では、現場で実践しながら進行形の創造プロセスとして現実を描き出す、一人称研究をとった。
これまで描かれてこなかった、身体運動表現の学び(創造プロセスとして)を描いた。

第一部は、著者自身が陸上競技(十種競技)の実践家として、「走り」という身体スキルを自らの現実を生きながら培ったプロセスを物語るものである。
はじめ、全身各部位や地面とに成り立つ多種多様な関係性へ志向していった。
ある怪我をきっかけとして、次第に、著者は自らの日常生活の動きや生活を取り巻くモノ(洗濯機や石ころやリュック)さえもが、
自らのスキルと密接に関わっていることを自覚し、実践に巻き込んでいった。
この過程にともなって、著者の走りの動作にも変化がみられた。

第二部では、実践家が「動く身体」を観るときに、その身体が醸し出す「表情」(=身体表情≠身体運動)を感得することを支援する
アプリケーションを制作した。感得した身体表情こそが、創造の源になり、実践家の学びを大いに促進しうるという仮説にもとづいている。
制作したアプリケーションをもちいて、学習実践をおこなった。
このアプリの設計には、第一部で著者自身が営んだスキル学習によって培ったものごとを色濃く反映している。

両部ともに一人称研究の方法論をとっている。
パラダイム転換しようと目論む。
随所で、神経科学的知見を参照し、その論拠としてもちいる。
現象学的な考え方をふまえて、いまだ認知についての学問領域には十分検討されていない「表情」をめぐる問題系について、
アプリとして「計算可能性と不可能性」を探究したことに価値がある。
現象学では通常、言葉をもちいて論じたり語る方法論をとる。
本研究のアプリで映像表現したりすることで、言葉だけでは迫り得なかったアクチュアリティに迫ることができた。
